\documentclass[12pt]{article}

\title{Term Project}
\author{Brian Camacho}


\usepackage{caption}
\usepackage{subcaption}
\usepackage{graphicx}
\usepackage{caption}
\usepackage{float}
\usepackage{amsmath}
	
\usepackage{color}

\newcommand{\source}[1]{\tiny\caption*{Source: {#1}} }
\newcommand{\Loss}{\mathcal{L}}
\begin{document}

\maketitle

\newpage
\pagenumbering{arabic}

\section{Literature Review}
The chapter begins with an overview of state of the art Facial Recognition methods, including
a brief explanation of general Computer Vision and Facial Recognition.
Then, Neural Network approximation and Artificial Intelligence acceleration techniques are
discussed, with a focus on Convolutional Neural Networks and Field-Programmable Gate Arrays. This
includes an  examination of benchmarking methods, and mostly discusses improvements to inference,
as opposed to training.

\subsection{Computer Vision}
Computer Vision, that is, using software to recognise objects in images, is a difficult
problem. Image data is unstructured, high-dimensional and subtle differences in pixel values
can completely change the object represented\cite{prince2012computer}.
Historically, distilling information from such data required experts to craft complex feature
extractors, which generalised poorly to a large number of classes.

\subsubsection{Convolutional Neural Networks}
The seminal 2012 paper by Kirzhevsky et al. used Convolutional Neural Network to achieve an mAP
of 16\% in the Imagenet Large Scale Visual Recognition Challenge, compared to the second-best
submission with an mAP of 26\%\cite{ILSVRC15}.
CNNs have since become the standard in computer vision tasks \cite{sze2017efficient}, including
Facial Recognition.
Usual network architectures start with a number of Convolutional, Pooling and Activation layers
which serve as feature extractors and turn a vector $v \in R^{W\times H}$, representing a
$W\times H$ image, into an internal vector embedding $x \in R^D$, where $D << WxH$, followed
by a final regression/classification layer.

\subsubsection{Facial Recognition}
If our goal is only recognising faces known at training time, Facial Recognition simplifies
into a classification problem. This is called closed-set FR.
The converse case is called open-set FR, where we work with identities which do not belong to
the training set.
Not only is open-set FR more challenging, but also more applicable in practice
\cite{liu2017sphereface}.
In normal classification, the learned vector embeddings need only to be separable by the final
classification (eg. softmax) layer.
Open-set FR, however, embedding to also be discriminative --- where we can expect the nearest
neighbour of an embedding to represent the same class. In other words, we need small intra-class
distances and large inter-class distances\cite{deng2019arcface}.
Parkhi et al\cite{parkhi2015deep} tackle this by treating initial CONV layers as a feature
extractor, and then replacing the softmax classifier head with a regression head based on
$(x_{Anchor}, x_{Positive}, x_{Negative})$ embedded face triplets, with loss $\Loss$ $$\Loss =
max(0, |x_{Anchor} - x_{Positive}| - |x_{Anchor} - x_{Negative}|)$$
Authors of SphereFace\cite{liu2017sphereface} take a different approach to the same problem,
by defining an Angular Softmax function, encouraging a discriminative distribution of classes,
with an adjustable margin hyperparameter $m$.
Given 2 weight vectors $W_1, W_2$, and the angle between them $\theta$, we can enforce an
angular margin of $\frac{m-1}{m+1}\theta$ \cite{li2018angular}.
Finally, Deng et. al synthesise different margin incorporating loss functions, producing
state of the art results with an additive (as opposed to multiplicative in SphereFace)
penalty\cite{deng2019arcface}.

\subsection{Deep Neural Network Acceleration}

\subsubsection{Training vs Inference}
Like all ML methods, DNNs have to be trained on data before they can be used in inference. Training
is composed of 2 stages --- Forward and Backward pass.
In forward pass, the input vector $x$ is transformed by layers $L_1, L_2,..., L_n$  in sequence,
then compared to target $y$ to compute loss $\Loss$:
\begin{gather}
\begin{split}
    o &= L_n( ... L_2(L_1(v)) ... )\\
    \Loss &= f(o, y)
\end{split}
\end{gather}
The symbolic gradients $\frac{\partial \Loss}{\partial W_i}$ for each layer weights $W_i$
are computed ahead of time as:
\begin{gather}
\begin{split}
    \frac{\partial \Loss}{\partial W_n} &= \frac{\partial f(o, y)}{\partial W_n}\\
    \frac{\partial \Loss}{\partial W_i} &= \frac{\Loss}{\partial L_{i+1}} * \frac{\partial
    L_{i+1}}{\partial W_i}
\end{split}
\end{gather}
Importantly, the symbolic gradients have to be evaluated at their respective inputs which,
in turn, need to be stored during forward pass.
In inference, however, only the forward pass is computed, and there is no need to store
intermediate input values or calculate loss, creating more opportunities for optimisation.
This work focuses on optimising the forward pass.


\subsubsection{CPU and GPU}
Most computation in Deep Neural Networks is expressed in terms of matricies of Floating Point
numbers, processed in multiply-and-accumulate operations.
These are trivially parallelisable, and can benefit from highly parallel compute
paradigms\cite{sze2017efficient}.
On both CPU and GPU, Single Instruction Multiple Data and Single Instruction Multiple Thread
can be leveraged via existing Basic Linear Algebra Subprograms libraries.
There also exist a number

\section{Problem Analysis}

\section{Design and Implementation}

\section{Evaluation}

\section{Conclusion}
\bibliography{bibliography}
\bibliographystyle{ieeetr}
\end{document}
