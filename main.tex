\documentclass[12pt]{article}

\title{Term Project}
\author{Brian Camacho}


\usepackage{caption}
\usepackage{subcaption}
\usepackage{graphicx}
\usepackage{caption}
\usepackage{float}
\usepackage{amsmath}

\usepackage{color}

\newcommand{\source}[1]{\tiny\caption*{Source: {#1}} }
\newcommand{\Loss}{\mathcal{L}}
\newcommand{\Fourier}{\mathcal{F}}
\begin{document}

\maketitle

\newpage
\pagenumbering{arabic}

\section{Literature Review}
The chapter begins with an overview of state of the art Facial Recognition methods, including
a brief explanation of general Computer Vision and Facial Recognition.
Then, Neural Network approximation and Artificial Intelligence acceleration techniques are
discussed, with a focus on Convolutional Neural Networks and Field-Programmable Gate Arrays. This
includes an  examination of benchmarking methods, and mostly discusses improvements to inference,
as opposed to training.

\subsection{Computer Vision}
Computer Vision, that is, using software to recognise objects in images, is a difficult
problem. Image data is unstructured, high-dimensional and subtle differences in pixel values
can completely change the object represented\cite{prince2012computer}.
Historically, distilling information from such data required experts to craft complex feature
extractors, which generalised poorly to a large number of classes.

\subsubsection{Convolutional Neural Networks}
The seminal 2012 paper by Kirzhevsky et al. used Convolutional Neural Network to achieve an mAP
of 16\% in the Imagenet Large Scale Visual Recognition Challenge, compared to the second-best
submission with an mAP of 26\%\cite{ILSVRC15}.
CNNs have since become the standard in computer vision tasks \cite{sze2017efficient}, including
Facial Recognition.
Usual network architectures start with a number of Convolutional, Pooling and Activation layers
which serve as feature extractors and turn a vector $v \in R^{W\times H}$, representing a
$W\times H$ image, into an internal vector embedding $x \in R^D$, where $D << WxH$, followed
by a final regression/classification layer.

\subsubsection{Facial Recognition}
If our goal is only recognising faces known at training time, Facial Recognition simplifies
into a classification problem. This is called closed-set FR.
The converse case is called open-set FR, where we work with identities which do not belong to
the training set.
Not only is open-set FR more challenging, but also more applicable in practice
\cite{liu2017sphereface}.
In normal classification, the learned vector embeddings need only to be separable by the final
classification (eg. softmax) layer.
Open-set FR, however, embedding to also be discriminative --- where we can expect the nearest
neighbour of an embedding to represent the same class. In other words, we need small intra-class
distances and large inter-class distances\cite{deng2019arcface}.
Parkhi et al\cite{parkhi2015deep} tackle this by treating initial CONV layers as a feature
extractor, and then replacing the softmax classifier head with a regression head based on
$(x_{Anchor}, x_{Positive}, x_{Negative})$ embedded face triplets, with loss $\Loss$ $$\Loss =
max(0, |x_{Anchor} - x_{Positive}| - |x_{Anchor} - x_{Negative}|)$$
Authors of SphereFace\cite{liu2017sphereface} take a different approach to the same problem,
by defining an Angular Softmax function, encouraging a discriminative distribution of classes,
with an adjustable margin hyperparameter $m$.
Given 2 weight vectors $W_1, W_2$, and the angle between them $\theta$, we can enforce an
angular margin of $\frac{m-1}{m+1}\theta$ \cite{li2018angular}.
Finally, Deng et. al synthesise different margin incorporating loss functions, producing
state of the art results with an additive (as opposed to multiplicative in SphereFace)
penalty\cite{deng2019arcface}.

\subsection{Deep Neural Network Acceleration}
\subsubsection{Roofline Model}
The Roofline Model is used in benchmarking deep neural networks in a way that generalised well
to real-world applications.
Attainable performance $P$ is expressed as $$P = min(\pi, \beta * I)$$ where $\pi$, $\beta$
and $I$ represent peak compute power, peak bandwidth and operational intensity, respectively.
RM accurately represents performanece being either compute- or memory-bound, where additional
compute power will not improve performance if data is already read from memory as fast as
possible, and vice versa.
Operational intensity represents how much computation is performed for each byte read from memory.
For instance by reusing parameters in different layers, we can improve performance even if the
process is memory bound.

\subsubsection{Training vs Inference}
Like all ML methods, DNNs have to be trained on data before they can be used in inference. Training
is composed of 2 stages --- Forward and Backward propagation.
In forward propagation, the input vector $x$ is transformed by layers $L_1, L_2,..., L_n$
in sequence,
then compared to target $y$ to compute loss $\Loss$:
\begin{gather}
\begin{split}
    o &= L_n( ... L_2(L_1(v)) ... )\\
    \Loss &= f(o, y)
\end{split}
\end{gather}
The symbolic gradients $\frac{\partial \Loss}{\partial W_i}$ for each layer weights $W_i$
are computed ahead of time as:
\begin{gather}
\begin{split}
    \frac{\partial \Loss}{\partial W_n} &= \frac{\partial f(o, y)}{\partial W_n}\\
    \frac{\partial \Loss}{\partial W_i} &= \frac{\Loss}{\partial L_{i+1}} * \frac{\partial
    L_{i+1}}{\partial W_i}
\end{split}
\end{gather}
Importantly, the symbolic gradients have to be evaluated at their respective inputs which,
in turn, need to be stored during forward propagation.
In inference, however, only forward propagation is computed, and there is no need to store
intermediate input values or calculate loss, creating more opportunities for optimisation.
This work focuses on optimising forward propagation.


\subsubsection{Vectorisation}
Most computation in Deep Neural Networks is expressed in terms of matricies of Floating Point
numbers, processed in multiply-and-accumulate operations.
These are trivially parallelisable, and can benefit from highly parallel compute
paradigms\cite{sze2017efficient}.
On both CPU and GPU, Single Instruction Multiple Data and Single Instruction Multiple Thread
can be leveraged via existing Basic Linear Algebra Subprograms libraries.

\subsubsection{Accelerating Convolution}
The initial convolutional layers use most resources (both memory and compute) in a CNN
\cite{karpathy2015cs231n},
making convolution the most lucrative target for optimisation.
Mapping the input matrix into a Toeplitz Matrix, and unwinding the convolutional filter into
a vector,
lets us represent the complex convolution operation as a much simple matrix vector
multiplication. This
comes at the expense of duplicating input data \cite{sze2017efficient}

The Convolution Theorem states that an elementwise multiplication of fourier-transformed kernel
and input is equivalent to a convolution of their non-transformed counterparts, more precisely:
\begin{gather}
	f * g = \Fourier^{-1}(\Fourier(f) \cdot \Fourier(g))
\end{gather}
where $*$ denotes convolution.
Computational complexity of "naively" convolving a $C \times W_i \times H_i$ input with a $C
\times W_k \times H_k$ kernel is $O(CW_iH_iW_kH_k)$.
Applying Fast Fourier Transform to the input, then doing elementwise multiplication followed
by inverse FFT, has complexity
$$O(CH_iW_ilog(W_iH_i) + 4CW_iH_i + W_iH_i)$$
Which represent the log-linear complixity of FFT, then the cost of piecewise multiplying 2
complex matricies,
then inverse FFT on the now 1-channel feature map. The speedup obtained from this approach
diminishes when $W_k, H_k << W_i, H_i$\cite{liu2016pruning}.
The Winograd convolution addresses this, outperforming FFT on small kernels.
The 2 approaches can be synthesised in a single network, applying Winograd/FFT to different
convolutional layers depending on kernel size \cite{zhuge2018face}

Spatial factorisation has been used to drastically improve efficiency of convolution kernels,
for instance the edge-detecting Sobel kernel.
An $W_k \times H_k$ kernel is factorised into a $1 \times W_k$ and $H_k \times 1$ vectors,
which are applied to the image in sequence.
For some kernels, this is $O(W_iH_i(W_k + H_k))$ operation is equivalent to the $O(W_iH_iW_kH_k)$
2D convolution.
Unfortunately, only a small subset of kernels are spatially separable (specifically, kernels
whose column vectors are all the same vector multiplied by a scalar).
In depthwise separable convolutions, we use a similar approach, where convolving a $W_i \times
H_i \times C$ image with $D$ kernels of size $W_k \times H_k \times C$ is decomposed into 2 stages.
First, each input channel is convolved with a $W_k \times H_l \times 1$.
The resulting intermediate feature map is then pointwise convolved with $D$ $1 \times 1 \times
C$ kernels.
Space and compute savings here stem from reusing the same $W_k \times H_k \times 1$ kernel for
all of the pointwise kernels.
This approach was introduced in \cite{howard2017mobilenets} and achieved state of the art results.

\subsubsection{Block Floating Point}
The IEEE 754 Floating Point representation uses 8 exponent bits to represent the position of
the binary point.
Instead of storing the exponent separately for each weight, they can be grouped by layer
\cite{courbariaux2014training}.

\subsubsection{Quantisation}
In training, most DCNNs use 32 Floating Point numbers to represent weights, activations and inputs.
Not all of this precision is needed in a forward pass, however.
In BinaryNet, for all layers except the first, weights and activation are quantised to $\{-1,
1\}$, represented with unset and set bits respectively.
Surprisingly, authors note only a small accuracy loss on small
networks\cite{courbariaux2016binarynet}, but a much larger (~30\%) on large networks
\cite{wang2019deep}.
When the accuracy loss is acceptable, this approach massively accelerates computation,
as the expensive FP32 matrix-vector product operations can be replaced with an XNOR and
POPCOUNT\cite{courbariaux2016binarynet}.
The benefits are especially visible in flexible architectures (ASIC/FPGA).
The accuracy loss can be compensated for by expanding the binarized network.
Umuroglu at al. speculate BNNs need to be expanded 2-11x to eliminate accuracy loss
\cite{umuroglu2016finn},
but use MNIST as the experimental dataset, which does not capture the more significant accuracy
loss for large networks.

Notably, binarized neurons are unable to deactivate. Ternarisation overcomes this by quantising
values to $\{-1, 0, 1\}$.
Another notable variation is Trained Ternary Weights, where we learn a weight $w$ from data,
and quantise to $\{-w, 0 ,w\}$.
This approach can naturally be extended to arbitrary-bitwidth quantisation.

In Fine Grained Quantisation, we choose different quantisation for different parts of the network.
Ideally, we could quantise each weight/activation with the best ratio of performance benefit
to accuracy loss,
but this significantly complicates the architectures.
Instead \cite{wu2016quantized} paritions the network into a number of quantisation domains
with kmeans.

Authors of LogNet\cite{lee2017lognet} use logarithmically quantised 4bit weights to drastically
outperform the usual, linearly quantised alternative.
This suggest the maximum expressible range of an activation is more important than precision
in preserving DCNN performance.

\subsubsection{Weight Reduction}
In all of machine learning, regularisation is commonly applied to models to both stop overfitting
and control computational cost.
Pruning, in the simplest case, is an extension of this technique, where a network is trained
with regularisation until some weights become 0, and then can be pruned away.
Authors of \cite{guo2016dynamic} achieve an impressive 17.7 times compression on AlexNet,
but the resulting network is much more complex, and in some cases even slower \cite{yu2017scalpel}.
Moreover, simple magnitude-based pruning does not account for interactions between different
weights.
Authors of \cite{yu2017scalpel} address the hardware implementation problems by taking the
target hardware parameters into account when pruning.
Limitations of magnitude-based pruning are addressed in \cite{yang2017designing} by overpruning
then re-introducing weights that reduce output error the most.

comment: I also want to discuss \cite{ullrich2017soft} here, which uses variational Bayesian learning
to synthesise the other compression approaches.
The paper itself is heavy on stats and information theory, and I need a lot more backgroud
reading for it, so I'll leave it out for now 
\section{Problem Analysis}

\section{Design and Implementation}

\section{Evaluation}

\section{Conclusion}
\bibliography{bibliography}
\bibliographystyle{ieeetr}
\end{document}
